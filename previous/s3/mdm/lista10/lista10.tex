\documentclass{article}
\usepackage[utf8]{inputenc}
\usepackage[english, polish]{babel}
\usepackage{forloop}
\usepackage[T1]{fontenc}
\usepackage{amsfonts}
\usepackage{tikz}
\usepackage{graphicx}
\usepackage{amsmath}
\usepackage{graphicx}
\title{Lista 10}
\author{Łukasz Magnuszewski}
\date{\vspace{-5ex}}
%\date{} 
\begin{document}
\maketitle
\section*{Zadanie 1}
abasdasdc
Najpierw na podstawie listy krawędzi utwórzmy listy sąsiędzedztwa dla obu grafów. 
Będzie to miało koszt $O(n+m)$. 

Następnie sąsiadów każdego wierzchołka sortujemy przy pomocy algorytmu sortowanie przez zliczanie. Sumaryczna liczba sortowanych wartości to $m$. Zaś liczba różnych możliwych wartości to wierzchołki w grafie czyli jest ich $n$. W takim razie wykonamy $O(m+n)$ operacji.

Mając posortowane listy sąsiedztwa możemy przeiterować się po wszystkich wierzchołkach i sprawdzić czy ich listy sądziedztwa dla grafu $G_1, G_2$ są takie same. Będzie to miało złożoność $O(n+m)$

Kazdy z kroków wykonujemy dokładnie raz, czyli sumarczyna złożoność będzie wynosić $O(n+m)$.


\section*{Zadanie 2}
\subsection*{podpunkt a}
Taki graf nie istnieje gdyż suma stopni wierzchołków jest nieparzysta, czyli nie spełnia lematu o uściskach dłoni.
\[
  1+2+2+3+3=11
\]


\subsection*{podpunkt b}
Wierzchołek piąty ma stopień 4, czyli jest połączony z każdym pozostałym wierzchołkiem. Zaś wierzchołki 1,2,3 mają stopień 1. Więc są połączone tylko z wierzchołkiem 5. Ale wierzchołek 4 musi być połączony z 3 wierzchołkami, a może być tylko z jednym(wierzchołek 5). CZyli taki graf nie istnieje.

\subsection*{podpunkt c}
Rozważmy możliwe podziały tego grafu dwudzielnego. I pokażmy że dla każdego podziału, nie działa taki ciąg stopni. Z tego wyniknie że podany graf nie istnieje.
\subsubsection*{5:0, 4:1}
Wtedy wierzchołki należące do lewej grupy mogą być połączone tylko z tymi należącymi do prawej grupy, a jest ich mniej niż 2. Czyli wierzchołki z lewej grupy nie mogłyby mieć stopnia 2.
\subsubsection*{3:2}
Każdy z lewych wierzchołków ma stopień 2. Czyli jest $3*2 = 6$ krawędzi idących z lewej do prawej strony. Nie da się tak podzielić 6 krawędzi między 2 wierzchołki tak by każdy miał stopień 2.
\subsubsection*{Pozostałe przypadki}
Pozostałe podziały są symetryczne do poprzednich przypadków.

\section*{Zadanie 4}
Weźmy wierzchołek o stopniu $n-2$ i nazwijmy go x. Jest on połaczyny z wszystkimi poza 1 wierzchołkiem nazwijmy go y. Jako że średnica grafu wynosi 2. to odległość między x i y wynosi 2. Weźmy wierzchołek leżący na tej ścieżce między x i y, nazwijmy go z. Mamy już przynajmniej $n-2 + 1$ krawędzi. Teraz rozważmy wiechołki które nie są x,y,z. Ich odległość od z wynosi maksymalnie 2. Czyli albo są połączone bezpośrednio z z. Albo z wierzchołkiem który jest połączony z(nie może to być x). Tych wierzchołków jest $n-3$ i każdy wymaga przynajmniej jednej krawędzi. Czyli w tym grafie muszą być przynajmniej $n-1 + n-3 = 2n-4$ krawędzie.

\section*{Zadanie 6}
Założmy niewprost że drogi łączące $a-b, c-d$ są rozłączne oraz że drogi łączące $a-c, b-d$. 

Wtedy Jak rozpatrzmy ścieżke $a-b-d-c$, to wiemy że ścieżka $a-c$ jest rozłączna z $b-d$. Czyli mamy 2 różne ścieżki idące między 2 wierzchołkami czyli mamy cykl. Co oznacza że ten graf nie jest drzewem.

\section*{Zadanie 11}
Rozważmy najpierw liczbę drzew o $n-1$ wierzchołkach. Jest ich z twierdzenia Caleya $(n-1)^{n-3}$. Teraz możemy rozważyć do którego wierzchołka podepniemy nasz wyróżniony wierzchołek. Czyli mamy $n-1$ opcji. Czyli mamy $(n-1)^{n-2}$ drzew o n wierzchołkach gdzie nasz wyróżniony wierzchołek jest liściem. Czyli prawdopodobieństwo tego wynosi
$\frac{(n-1)^{n-2}}{n^{n-2}} = (\frac{n-1}{n})^{n-2}$
Policzmy teraz granice tego w nieskonczoności.
\[
  \lim_{n \to \infty} (\frac{n-1}{n})^{n-2} = \lim_{n \to \infty} ((1 + \frac{1}{-n})^{-n})^{\frac{n-2}{-n}} = \frac{1}{e}
\]

\section*{Zadanie 3}
\subsection*{Graf niespójny}
Jeśli graf jest niespójny to odległość w dopełenieniu grafu wynosi maksymalnie 2. Jeśli 2 wierzchołki są w różnych spójnych to jest między nimi krawędź. Jeśli są w tej samej. To najpierw przeskakujemy do innej spójnej i potem wracamy do drugiego wierzchołka.
\subsection*{Graf spójny}
Weźmy te wierzchołki x,y dla których $d(x,y) = d(G) > 3$
Teraz ustalmy dowolny wierzchołki a,b. Jeśli nie ma między nimi krawędzi w G, to jest między nimi krawędź w G' czyli $d'(a,b) = 1$. Rozpatrzmy od tego momementu niepołączone a,b. W takim razie weźmy wierzchołek c dla którego $d(a, c) > 1 \and d(b, c) > 1$ W takim razie w G' istnieje krawędź między a,c oraz b,c. Czyli mamy ścieżkę o długości 2 $a-c-b$. czyli $d(a,c) = 2 < 3$
\subsubsection*{Istnienie c}
Załóżmy niewprost że taki wierzchołek nie istnieje. Wtedy $\forall x: x \neq a \wedge x \neq b \implies d(x,a) = 1 \vee d(x, b) = 1$. Bez straty ogólnośc $d(a, x) = 1, d(b,y) = 1 \vee d(a,x) = 1 = d(a,y) $. Wtedy w pierwszym przypadku lub mamy ścieżkę $x-a-b-y$ czyli $3 < d(x,y) = 3$. Zaś w drugim przypadku lub, mamy ścieżkię $x-a-y$ czyli $3 < d(x,y) = 2$. Sprzeczność!
\end{document}



