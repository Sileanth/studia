\documentclass{article}
\usepackage[utf8]{inputenc}
\usepackage[legalpaper, margin=1in]{geometry}
\usepackage[T1]{fontenc}
\usepackage[polish]{babel}
\usepackage[utf8]{inputenc}
\usepackage{tikz}
\usetikzlibrary{automata,positioning} % LaTeX and plain TeX

\usepackage{amsmath}
\newcommand{\Lg}{{\Sigma^*}} 

\begin{document}


\section*{Zadanie 1}
\subsection*{Treść}
Rozważmy język $L = \{w0s : |s| = 9\}$, złożony z tych słów nad alfabetem $\{0, 1\}$
których dziesiąty symbol od końca to 0. Udowodnij, że DFA rozpoznający ten język ma co
najmniej 1024 stany.
\subsection*{Rozwiązanie}
Skorzystajmy z twierdzenia o indeksie, wskazując rodzinę słów o mocy 1024, z których każde jest w innej klasie abstrakcji $\sim_L$.

Niech $R$ będzie zbiorem liczb od 0 do 1023 włącznie, zapisanych binarnie przy pomocy 10 bitów (pozwalamy na wiodące zera).

Weźmy dowolne $w_1, w_2 \in R$. Jako że odpowiadają one różnym liczbom to istnieje bit $k$ taki że $w_1 [ k ] \neq w_2[ k ]$.
Bez straty ogólności $w_1 [ k ] = 1$ oraz $ w_2 [ k ] = 0$.

Wtedy $w_1 0^{10 - k} \in L$ i $w_2 0^{10 - k} \notin L$, z czego wynika że dowolne 2 słowa z $R$ są w innej klasie abstrakcji, czyli
automat rozpoznający $L$ musi mieć przynajmniej $|R| = 1024$ stanów.

\section*{Zadanie 4}
\subsection*{Treść}
(za 2 punkty) Dla danego języka $L \subseteq L^*$ przez $L^*$ rozumiemy najmniejszy język
spełniający następujące warunki:
\begin{itemize}
    \item $\epsilon \in L^*$
    \item $\forall x, y. [ x \in L^* \land y \in L ] \implies  x y \in L^*$  
\end{itemize}
Gdzie $\epsilon$ oznacza, jak zawsze, słowo puste.
Niech L będzie dowolnym podzbiorem $\{ 0 \}^*$. Udowodnij, że $L^*$ jest językiem regularnym.

\subsection*{Rozwiązanie}
Mamy tutaj do czynienia z unarnym alfabetem, więc od tego momentu słowa utożsamiam z ich długością. 
Możemy to zrobić bo liczby naturalne są wolnym monoidem nad $1$.



Rozwiążmy najpierw podprzypadek i potem uogólnijmy go na całość.

\subsubsection*{Podprzypadek}
Załóżmy że istnieje $p, q \in L^*$ takie że $p$ i $q$ są względnie pierwsze. Wtedy z rozszerzonego algorytmu euklidesa
otrzymujemy $x, y$ spełniające $x p + y q = \gcd(p, q) = 1$ (bo $p, q$ względnie pierwsze). 

Możemy w takim razie otrzymać też wszystkie liczby od $1$ do $ p q$.
\[
    n (x p + y q) = n
\]
Czyli otrzymaliśmy wszystkie możliwe reszty z dzielenia przez $p q$, oraz dodatkowo możemy się przesuwać o $p q$ do przodu dodając p q razy.
Czyli od pewnego momentu wszystkie liczby należą do języka.

Problem jest tylko taki że $x,y$ mogą być ujemne, co nie pasuje naszej interpretacji. Ale zauważmy że jeśli dodamy p q razy, oraz q
p razy, to reszta z dzielenia się nie zmieni, a współczynniki urosną.

Istnieje takie $z$, że dla każdego $n$, $n x \leq z q$ oraz $n y \leq z p$.

Czyli od liczby $ z x p + z y q$ wszystkie liczby należą do języka. A wszystkie poprzednie liczby możemy zaifować.


Skonstruujmy teraz żuczka by był zawsze szczęsliwy i jak najbardziej wybredny \newline

% TODO, nie potrafie w tikz
 \begin{tikzpicture}[shorten >=1pt,node distance=2cm,on grid,auto]
    \draw[help lines] (-3,0) grid (8,5);
  
    \node[state,initial]  (q_0)                      {$q_0$};
    \node[state]          (q_1) [ right=of q_0] {$q_1$};
    \node[state]          (q_2) [ right=of q_1] {$q_2$};
    \node[state,accepting](q_3) [ right=of q_2] {$q_3$};
  
    \path[->] (q_0) edge              node        {0} (q_1)
              (q_1) edge              node        {0} (q_2)
              (q_2) edge              node        {0} (q_3);
            
  \end{tikzpicture}


\subsubsection*{Uogólnienie}
Jeśli nie zachodzi podprzypadek, to istnieje $v = \gcd(L*)$. To się sprowadza do poprzedniego przypadku, 
tylko żuczek w każdym kroku leci o $v$ pól do przodu.


\end{document}