
\documentclass{article}
\usepackage{./../../templatka}
\author{Łukasz Magnuszewski}

\title{Tutorial 5}

\begin{document}
\maketitle
In all exercises, $\Sigma$ is a linearly ordered alphabet.
\section*{Exercise 1 (2 points)}
Prove that $\leq_{slex}$ is a well ordered over $\Sigma^*$. Is $\leq_{lex}$ also a well-ordered over $\Sigma^*$?

\subsection*{slex}
Take any non empty language $L \subseteq \Sigma^*$. Then there is a word with shortest length $n$.
We can take now $S = \{ w \in L | |w| = n \}$. $S$ is finite because $\Sigma$ is finite. $|S| \leq |\Sigma|^n$
Every finite, linear order is well ordered, so there is minimal element in $S$. Which is also minimal element in $L$.

\subsection*{lex}
Consider $\Sigma = \{a, b \}$ with order $a < b$ and $L = \{w \in \Sigma^* : |w|_b = 1 \land b(|w|)  \}$(words with exactly one b at the end). $L = \{b, ab, aab, aaab, \ldots \}$.

Suppose there exists minimal $w \in L$. Then we can add a at the start of w and $aw < w$. Contradiction!

\section*{Exercise 2}
Here, $\Sigma$ is the binary alphabet $\{0, 1\}$, (with its natural order). 
Considering the shortlex order, what is then nth word of $\Sigma^*$ (the empty word being the 0th word).

We can define isomorphism beetwen $\Sigma^* \cong \omega \backslash \{0\} \cong \omega$. 
$f(\epsilon) = 1$, $f(0 * w) = 2 * f(w), f(1 * w) = 2 * f(w) + 1$. So then nth word is $f^{-1}(n+1)$.


\section*{Exercise 3}
$\Sigma$ is posssibly infinite alphabet, which is well ordered. Prove that for every $n \in \mathbb{N}$, $\leq_{lex}$ is a well ordered over the language $L = \Sigma^{\leq n}$ of words over $\Sigma$ of length $\leq n$.


Take any $S \subseteq L$. If s contains $\epsilon$ then it is the minimal element. So from now we assume that $S$ dont have empty word.

We can take set of first letters of words in $S$. Lets name it $F$. $\Sigma$ is well ordered and $F \subseteq \Sigma$ so there is minimal element $m \in F$.
Now let's take $S' = \{ w \in S : \exists r. w = m * r \}$. Every element in $S'$ is smaller than elements in $S \backslash S'$. So minimal element can only be in $S'$. If there are only words of length 1 in $S'$ then we can use well ordering of $\Sigma$ to find minimal element or if there is an empty word, then that word in minimal. If not we can reapet this process(Each step reduce length by 1, so there can be maximal n steps, so proccess will always finish).

\section*{Exercise 4}
Here, $\Sigma$ is the unary alphabet $\{a \}$. 
Draw the part of Hankel matrix for the length series
$w \to |w|$. Exhibit a basis among the lines, and prove that it is actually a basis.

\begin{matrix}
  & \epsilon & a^1 & a^2 & a^3 & a^4 & a^5 & \ldots \\
  \epsilon & 0 & 1 & 2 & 3 & 4 & 5 & \ldots \\  
  a^1 & 1 & 2 & 3 & 4 & 5 & 6 & \ldots \\  
  a^2 & 2 & 3 & 4 & 5 & 6 & 7 & \ldots \\  
  a^3 & 3 & 4 & 5 & 6 & 7 & 8 & \ldots \\  
  a^4 & 4 & 5 & 6 & 7 & 8 & 9 & \ldots \\  
  a^5 & 5 & 6 & 7 & 8 & 9 & 10 & \ldots \\
  \ldots & \ldots & \ldots &\ldots &\ldots & \ldots &\ldots & \ldots \\
\end{matrix}

Basis of this matrix is $\{R_\epsilon, R_a \}$. 


Proof that these rows generate others.
$R_a - R_\epsilon = [1 , 1, 1, \ldots]$ is a row full of ones.
Then we can express $R_{a^n} = n * (R_a - R_\epsilon) + R_\epsilon$ 


Proof that they are lineary independent. We can't express $R_a$ in terms of $R_\epsilon$ because
there is no such $x$ that $x * 0 = 1$. So $x * R_\alpha \neq R_a$(they will be diffrent on 0th column.

\section*{Exercise 5}
Almost the same as exercise 4. Only that there are duplicated rows.

\section*{Exercise 6}
Wymaga poprawki
For simplicity consider an singleton alphabet $\Sigma = \{a \}$.
Each row can be interprted as below discrete function
\[ f_k(x) = (k+x)^{k+x}\] 
Lets show that this matrix has infinite rank.
Suppose there is a finite basis $B = \{ R_{a_1}, R_{a_2}, \ldots, R_{a_l} \}$.
Then we can take $k = (\sum_{i=1}^k a_i) + 1000$. Then $O(f_k(x))$ is larger than any linear combination of basis.

\end{document}
