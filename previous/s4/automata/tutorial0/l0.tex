\documentclass{article}
\usepackage{./../aut}
\title{Tutorial 1}
\author{Łukasz Magnuszewski}


\begin{document}
\maketitle

\section*{Exercise 2}
\[
  c^*
  (
    ((a \cup c)^*  b (a \cup c)^* b (a \cup c)^*)^*
    \cap 
    ((b \cup c)^*  a (b \cup c)^* a (b \cup c)^* a (b \cup c)^*)^* 
  )
  c^*
\]

\section*{Exercise 5}
First let's look at class of 2. Its trivial that 2 is not in relation with any other prime.
So from this moment we consider only primes greater than 2.

Lets proof that each prime is in diffrent class of abstraction. We will proove that by contradiction
Assume that there exists primes $p,q > 2$ such that $p \sim q$. Without loss of generality
$p < q$ and $q-p = x$. 

There exits prime $r > p,q$, then $x \equiv x \mod r$. Then $q + (r + x - q)$ is a prime so from assumption
$p + (r  - q)$ is also prime, which is equal $k r + x - x = k r$ so is not prime.
Contradiction! 


So each prime has its own class, so there is infitly many class of abstraction, so from Myhill-Nerode theorem, 
the language isnt regular.


\section*{Exercise 7}
Consider regular language $L$ generated by following regular expression $(ab)^*$. Then $L!$ 
is set of words with equal number of a and b. 

Proof by contradiction. Suppose that $L!$ is regular, so take $N$ from pumping lema. Then consider word 
$a^n b^n = x y z$ such that $|x y| \leq n|$, so y consists only of b. Then $x y^4 z \not \in L!$ 
beacuse there is more a than b. Contradiction!

\end{document}