\documentclass{article}
\usepackage[utf8]{inputenc}
\usepackage[english, polish]{babel}
\usepackage{forloop}
\usepackage[T1]{fontenc}
\usepackage{amsfonts}
\usepackage{tikz}
\usepackage{graphicx}
\title{Lista 4}
\author{Łukasz Magnuszewski}
\date{\vspace{-5ex}}
%\date{} 
\begin{document}

\maketitle

\section*{Zadanie 1}
\subsection*{Treść}
Niech $ \{ P_k \}$ będzie ciągiem standardowych wielomianów ortogonalnych w przedziale $[a, b]$, z wagą $p(x)$. Wykazać że zachodzi związek rekurencyjny
\[ P_0(x) = 1, \; P_1(x) = x - c_1, \]
\[ P_k(x) = (x - c_k) P_{k-1}(x) - d_k P_{k-2}(x) \; (k = 2,3, \dots) \]
gdzie
\[ c_k = \frac{ \langle x P_{k-1}, P_{k-1}  \rangle}{\langle P_{k-1}, P_{k-1} \rangle } \] 
\[ d_k = \frac{ \langle  P_{k-1}, P_{k-1}  \rangle}{\langle P_{k-2}, P_{k-2} \rangle } \] 

\subsection*{Rozwiązanie}
Przeprowadźmy silną indukcję po k:


\subsubsection*{Baza indukcji}
Dla $k = 0$ w oczywisty sposób zachodzi zależność. Rozpatrzmy więc $k = 1$. Jako że $P_0 \perp P_1$ to:

\[ 
    0 = \langle P_0 , P_1 \rangle=  \langle 1, x - c_1 \rangle   = \langle 1 , x \rangle - c_1  \langle 1, 1 \rangle
\]
Więc 
\[
    c_1 = \frac{\langle x , 1 \rangle}{\langle 1,1 \rangle } =    \frac{\langle P_0 x , P_0 \rangle}{\langle P_0,P_0 \rangle } 
\]

\subsubsection*{Krok indukcyjny}
Pokażmy że ta zależność zachodzi dla k wielomianu. Rozpiszmy $P_k$ w następujący sposób

\[
   P_k = x P_{k-1} + \sum_{i=0}^{k-1} \alpha_i P_i
\]
Teraz wiemy że $P_k$ ortogonalne z poprzednimi wielomianami. Rozpatrzmy $j \leq k-3 $

\[
    \langle P_k , P_j \rangle = \langle x P_{k-1} + \sum_{i=0}^{k-1} \alpha_i P_i , P_j \rangle = \langle x P_{k-1} , P_j \rangle + \sum_{i=0}^{k-1} \alpha_i \langle P_i , P_j \rangle 
\]
Korzystając z tego że $ \{ P_x \} $ to ciąg ortogonalny, otrzymujemy
\[
    \langle x P_{k-1} , P_j \rangle + \alpha_j \langle P_j, P_j \rangle   
\]
Udowodnijmy teraz krótki lemat 1
\[ 
    \langle x f , g \rangle = \int_b^a p(x) x f (x) g(x) dx = \int_b^a p(x) f x (x) g(x) dx = \langle f , x g \rangle 
\] 
Używając tego lematu otrzymujemy
\[
     \langle x P_{k-1} , P_j \rangle + \alpha_j \langle P_j, P_j \rangle = \langle  P_{k-1} , x P_j \rangle + \alpha_j \langle P_j, P_j \rangle 
\]
Wielomian $x P_j$ ma mniejszy stopień niż $P_{k-1}$ więc są one ortogonalne. Czyli wychodzi
\[
   0 = \langle P_k , P_j \rangle = \alpha_j \langle P_j, P_j \rangle 
\]
Wielomian $P_j$ nie jest zerowy więc jego iloczyn skalarny z samym sobą jest niezerowy więc $\alpha_j = 0$. Usuwając zera z sumy w wzorze na $P_k$ wychodzi
\[ P_k = x P_{k-1} + \alpha_{k-1} P_{k-1} + \alpha_{k-2} P_{k-2}\]

Rozpatrzmy teraz $j = k-2$
\[ 
    \langle P_k , P_{k-2} \rangle = \langle x P_{k-1} + \alpha_{k-1} P_{k-1} + \alpha_{k-2} P_{k-2} , P_{k-2} \rangle 
\]
Można to porozbijać z liniowości
\[
    \langle x  P_{k-1}, P_{k-2} \rangle + \langle \alpha_{k-1} P_{k-1}, P_{k-2} \rangle +\alpha_{k-2}  \langle P_{k-2} , P_{k-2} \rangle 
\]
Środkowy wyraz się zeruje z ortogonalności
\[
    \langle x P_{k-1}, P_{k-2}  \rangle + \alpha_{k-2} \langle  P_{k-2} , P_{k-2} \rangle = 0
\]
Korzystając z lematu 1 wychodzi wzór na $\alpha_{k-2}$
\[
    \alpha_{k-2} = - \frac{\langle  P_{k-1}, x P_{k-2}  \rangle}{\langle  P_{k-2} , P_{k-2} \rangle }
\]
Mamy już prawie dobrą postać, ale trzeba naprawić licznik, rozpiszmy w tym celu $P_{k-1}$ podobnie jak rozpisaliśmy $P_{k}$
\[ P_{k-1} - x P_{k-2}  =  \sum_{i-0}^{k-2} \beta_i P_i \]
Czyli wychodzi 
\[ x P_{k-2}  =P_{k-1} - \sum_{i-0}^{k-2} \beta_i P_i \]
Możemy to podstawić
\[
    \langle  P_{k-1}, x P_{k-2}  \rangle = \langle P_{k-1} , P_{k-1} - \sum_{i-0}^{k-2} \beta_i P_i \rangle = 
\]
Z liniowości wychodzi
\[
    \langle P_{k-1} , P_{k-1} - \sum_{i-0}^{k-2} \beta_i P_i \rangle = 
    \langle   P_{k-1} , \rangle + \sum_{i-0}^{k-2} \beta_i \langle  P_{k-1}, P_i \rangle
\]
Z ortogonalności $P_i \perp P_{k-1}$ wychodzi
\[
    \langle  P_{k-1}, x P_{k-2}  \rangle = \langle   P_{k-1} ,P_{k-1} \rangle + \sum_{i-0}^{k-2} \beta_i \langle  P_{k-1}, P_i \rangle = \langle P_{k-1}, P_{k-1}  \rangle
\]
Podstawiając to do wzoru na $\alpha_{k-2}$ wychodzi
\[
    \alpha_{k-2} = - \frac{\langle  P_{k-1},  P_{k-1}  \rangle}{\langle  P_{k-2} , P_{k-2} \rangle }
\]
Co zgadza się z wzorem na $d_k$.

Rozpatrzmy teraz $j = k-1$. Możemy też rozpisać $P_k$
\[
    \langle P_k , P_{k-1} \rangle =   \langle x P_{k-1} + \alpha_{k-1} P_{k-1} + \alpha_{k-2} P_{k-2}, P_{k-1} \rangle 
\]
Z liniowości wychodzi
\[
    \langle P_k , P_{k-1} \rangle =   \langle x P_{k-1} , P_{k-1} \rangle +
    \langle \alpha_{k-1} P_{k-1},  P_{k-1}  \rangle +
    \langle \alpha_{k-2} P_{k-2} ,  P_{k-1} \rangle
\]
Z $P_{k-2} \perp P_{k-1}$ wychodzi
\[
    \langle P_k , P_{k-1} \rangle = 0 =   \langle x P_{k-1} , P_{k-1} \rangle +
    \langle \alpha_{k-1} P_{k-1},  P_{k-1}  \rangle 
\]
Wychodzi więc następujący wzór na $\alpha_{k-1}$
\[
    \alpha_{k-1} = - \frac{\langle x P_{k-1} , P_{k-1} \rangle}{\langle P_{k-1},  P_{k-1}  \rangle }
\]
Czyli wychodzi dokładnie $-c_k$ Podstawiając alphy do wzory na $P_k$ wychodzi
\[ 
    P_k = x P_{k-1} + \alpha_{k-1} P_{k-1} + \alpha_{k-2} P_{k-2} =
    P_{k-1} (x -
    \frac{\langle x P_{k-1} , P_{k-1} \rangle}{\langle P_{k-1},  P_{k-1}  \rangle })  + \frac{\langle  P_{k-1},  P_{k-1}  \rangle}{\langle  P_{k-2} , P_{k-2} \rangle } P_{k-2} = P_{k-1} (x - c_k) - d_k P_{k-2}
\]
\section*{Zadanie 2}
Niech $\overline{T}_k(x) $ będzie standardowymi wielomianami ortogonalnymi w przedziale $[-1, 1]$ z wagą $(1-x^2)^{-\frac{1}{2}}$. Znaleźć związek rekurencyjny spełniany przez te wielomiany

Zgadujemy na podstawie zadania 1 z listy 7, że ten ciąg wielomianów to podkręcone wielomiany Czybeszewa. Czyli są spełnione następujące własności
\[ \overline{T}_0(x) = 1 \qquad \overline{T}_1(x) = x \]
\[ \overline{T}_k(x) = \overline{T}_{k-1}(x) - \frac{1}{2} \overline{T}_{k-2}(x)  = 2^{-(k-1)} T_k(x)\]

Z zadania l7.1 wiemy że dla $k \neq l$
\[
  \langle T_k , T_l \rangle = \int^a_b (1-x^2)^{(-1/2)} T_k(x) T_l(x) dx= 0
\]
W takim razie
\[
    \langle \overline{T}_k \overline{T}_l \rangle = \langle T_k 2^{-(k-1)}, T_l 2^{-(l-1)}\rangle =
    \int^a_b (1-x^2)^{-1/2} T_k(x) 2^{-(k-1)} T_l(x) 2^{-(l-1)} dx
\]
Wyciągając stałe przed nawias wychodzi
\[
    2^{-(k-1)} 2^{-(l-1)} \int^a_b (1-x^2)^{-1/2} T_k(x)  T_l(x)  dx = 0 = \langle \overline{T}_k \overline{T}_l \rangle
\]
Czyli rzeczywiście nasz zgadnięty ciąg jest standardowym ciągiem wielomianów, dla tego przedziału i wagi.
\subsection*{Zadanie 3}
Jakim wzorem wyraża się n-ty wielomian optymalny dla funkcji $f$ w sensie normy
\[
  ||f||_2 := \sqrt{\int^1_{-1} 
    (1-x^2)^{-1/2} f^2 (x) dx
  }  
\]
Zauważmy że jest to taka sama norma jak w poprzednim zadaniu, więc skorzystajmy z wzoru na wykładzie dla ortogonalnego ciągu $\overline{T}_k$
\[
  w^*_n = \sum^n_{k=0} \frac{\langle f, P_k \rangle}{\langle P_k , P_k \rangle} P_k  
\]
Dowód:
analogicznie do zadania 7 z poprzedniej listy
Korzystając z zadania 5 z poprzedniej listy wystarczy pokazać że
\[ 
    \forall w \in \Pi_n \langle f - w^*_n , w \rangle = 0 
\]
Możemy sie ograniczyć do wektorów bazowych, bo każdy wektor można przedstawić jako kombinację liniową wektorów bazowych. Rozważmy jako bazę nasz ciąg ortogonalny obcięty do n.
Wtedy korzystając z liniowości wychodzi
\[
  \langle f -  \sum^n_{k=0} \frac{\langle f, P_k \rangle}{\langle P_k , P_k \rangle} P_k, P_j \rangle = \langle f, P_j \rangle - \sum^n_{k=0} \frac{\langle f, P_k \rangle}{\langle P_k , P_k \rangle}\langle  P_k, P_j \rangle
\]
Korzystając że $P_k \perp P_l$ dla $k \neq l$ wychodzi
\[
  \langle f, P_j\rangle - \frac{\langle f , P_j \rangle}{\langle P_j , P_j \rangle} \langle P_j, P_j \rangle = 0
\]
\section*{Zadanie 4}
Niech $p_n, q_n \in \Pi_n$ będą wielomianami optymalnymi dla funkcji ciągłej na odcinku $[a,b]$ w sensie normy jednostajnej. Udowodnić że $p_n \equiv q_n$. Co z tego wynika?

Dla $n =1$ trywialne.
Od tego momentu rozpatrujemy $n > 1$
Z twierdzenia Czybeszewa o Alternansie. Istnieją takie punkty $x_i,y_i$ dla $i \leq n+1$
Że zachodzą następujące własności
\[ f(x_i) - p_n(x_i) = (-1)^i sp E \qquad f(y_i) - q_n(y_i) = (-1)^i sq E \]



\end{document}



