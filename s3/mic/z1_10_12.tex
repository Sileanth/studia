\documentclass{article}
\usepackage[utf8]{inputenc}
\usepackage[english, polish]{babel}
\usepackage{forloop}
\usepackage[T1]{fontenc}
\usepackage{amsfonts}
\usepackage{tikz}
\usepackage{graphicx}
\usepackage{amsmath}
\title{Zadanie 1.10.12}
\author{Łukasz Magnuszewski}
\date{\vspace{-5ex}}
%\date{} 
\begin{document}

\maketitle
\section*{Treść}
Udowodnić że jeśli A jest nieskończonym $\sigma$-ciałem to A ma przynajmniej  $\mathfrak{c}$
elementów
\section{Rozwiązanie} 
Rozważamy sigma-ciało rozpiętę nad jakąś przestrzenią, oznaczmy ją jako V.
Weźmy $B \subseteq A$ taki że $|B| = \aleph_0$.

Przyjmijmy następujące oznaczenia 
\[ 
    C_x = (\bigcap \{ X \in B : x \in X \})  \cup (\bigcap \{ X \in B : x \not\in X \})
\]
Zauważmy że dla każdego x: $C_x$ powstaje w przeliczalnej liczbie operacji mnogościowych ze zbiorów należących do A, czyli każdy taki zbiór należy do A.

Rozważmy teraz rodzinę $C = \{ C_v : v \in V\}$. Zauważmy że zbiory z tej rodziny są parami rozłączne. Oraz że każdy zbiór z B da się rozpisać w następujący sposób $b = \bigcup_{x \in b} C_x$. Czyli w takim razie $|A| \geq |C| \geq |B| = \aleph_0$.

Rozważmy teraz $E = \{ \bigcup x : x \in P(C) \} $. Jako że C jest rodziną zbiorów rozłącznych parami, to $|E| = |P(C)|$ czyli continuum. Oraz $E \subseteq A$ czyli moc A wynosi conajmniej continuum.



\end{document}