\documentclass{article}
\usepackage[utf8]{inputenc}
\usepackage[english, polish]{babel}
\usepackage{forloop}
\usepackage[T1]{fontenc}
\usepackage{amsfonts}
\usepackage{tikz}
\usepackage{graphicx}
\usepackage{amsmath}
\title{Zadanie 1.11.F}
\author{Łukasz Magnuszewski}
\date{\vspace{-5ex}}
%\date{} 
\begin{document}

\maketitle
\section*{Treść}
Przeprowadzić następującą konstrukcję zbiori Vitali'ego: Dla $x,y \in [0,1)$,
niech $x \sim  y \iff x - y \in \mathbb{Q}$. Sprawdzić że $\sim$ jest relacją równoważności. Niech Z będzie zbiorem który z każdej klasy abstrakcji tej relacji wbyiera dokładnie jeden element. Sprawdzić że $\bigcup_{q \in \mathbb{Q}} (Z \oplus q) = [0,1)$, gdzie $ \oplus$ oznacza dodawanie mod 1.

\section*{Rozwiązanie}
\subsection*{Relacja równoważności}
Sprawdźmy najpierw, czy $\sim$ jest relacją równoważności.
\subsubsection*{Symetryczność}
Ustalmy dowolne a,b takie że $a \sim b$ wtedy $\exists q \in \mathbb{Q}, a - b = q$.
Ale wtedy $b - a = -q \in \mathbb{Q}$ czyli $b \sim a$.

\subsubsection*{Zwrotność}
Astalmy dowolne a, wtedy $a-a = 0 \in \mathbb{Q}$ czyli $ a \sim a$


\subsubsection*{Przechodniość}
Ustalmy dowolne a, b, c takie że $a \sim b \sim c$. Wtedy $\exists p,q \in \mathbb{Q}$ że $a-b= p, b-c = q$ Wtedy $p - q = a-b - (b-c) = a - c$. Czyli $a-c \in \mathbb{Q}$ czyli $a \sim c$.

\subsection*{Sprawdzenie}

Weźmy selektor z rodziny klas abstrakcji $\sim$, nazwijmy go Z.
Sprawdźmy teraz czy $\bigcup_{q \in \mathbb{Q}} (Z \oplus q) = [0,1)$

\subsubsection*{$\subseteq$}
Ustalmy dowolny $x \in \bigcup_{q \in \mathbb{Q}} (Z \oplus q)$ wtedy $\exists q,x \in Z \oplus q$ Czyli $\exists z \in Z \subseteq [0, 1), q \oplus z = x$ Wtedy $q \oplus z \in [0, 1)$ z własności dodawania modulo 1.     
\subsubsection*{$\supseteq$}
Ustalmy dowolny $x \in [0, 1)$ Weźmy jego reprezentanta i oznaczmy go jako z. Jako że $z \sim x$ to $\exists q, x = z + p = z \oplus p \in Z$ Czyli $x \in \bigcup_{q \in \mathbb{Q}} (Z \oplus q)$

\subsection*{Niemierzalność Z}
\subsubsection*{Lemat}
Pokażmy że następująca suma jest rozłączna:
$\bigcup_{q\in \mathbb{Q} \cap [0,1)} Z \oplus q$
Załóżmy niewprost że $\exists q,p \in \mathbb{Q}: q \neq p$ takie że $Z \oplus p \cap Z \oplus q \neq \emptyset$. Wtedy $\exists a: a \in Z \oplus p$ oraz $a \in Z \oplus q$. Ale to oznaczna że $\exists z_p, z_q \in Z$ takie że $z_p \oplus p = x = z_q \oplus q$. Ale to by znaczyło że $z_p = z_q$. Czyli $p=q$, co oznacza sprzecznośc i kończy dowód lematu. 


\subsection*{Dowód niewprost}
Załóżmy niewprost że Z jest mierzalne w sensie Lebesguea. Ustalmy dowolne $q \in \mathbb{Q} \bigcap [0,1)$. Wtedy $\lambda (Z \oplus q) = \lambda(Z)$ z niezmienniczności miary Lebesguea. Korzystając z lematu:
$\lambda([0, 1)) = \lambda ( \bigcup_{x\in \mathbb{Q} \cap [0,1)} Z \oplus q ) = \sum_{x\in \mathbb{Q} \cap [0,1)} \lambda(Z \oplus q) = \sum_{x\in \mathbb{Q} \cap [0,1)} \lambda(Z)$ Wtedy jeśli $\lambda(Z) > 1$ to cała suma wynosi nieskończoność. A w przeciwnym wypadku 0. A powinno wyjść 1. Czyli mamy sprzeczność z której wynika że Z nie jest mierzalne.

\end{document}



