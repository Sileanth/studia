\documentclass{article}
\usepackage[utf8]{inputenc}
\usepackage[english, polish]{babel}
\usepackage{forloop}
\usepackage[T1]{fontenc}
\usepackage{amsfonts}
\usepackage{tikz}
\usepackage{graphicx}
\usepackage{amsmath}
\title{Zadanie 2.6.D}
\author{Łukasz Magnuszewski}
\date{\vspace{-5ex}}
%\date{} 
\begin{document}

\maketitle
\section*{Treść}
Niech $f : \mathbb{R} \to \mathbb{R}$ będzie dowolną funkcją spełniają warunek $f(x+y) = f(x) + f(y)$. Sprawdzić wtedy że $f(x) = a x$ dla wszystkich $q \in \mathbb{Q}$ gdzie $a = f(1)$.

\section*{x wymierne}
Rozpatrzmy trzy przypadki
\subsection*{x = 0}
Wtedy $f(0) = f(0 + 0) = f(0) + f(0) = 2 f(0)$ Czyli $f(0) = 0 = 0 * a $.
\subsection*{x > 0} 
Rozważmy najpierw x naturalne. $f(x) = f(x * 1) = f(1 + 1 + \dots) = \sum_{i=1}^x f(1) = x a$

Teraz rozważmy x mniejsze od 1. $x = \frac{p}{q}, p,q \in \mathbb{N}$. W takim razie \[
  a p = f(p) = \sum^q_{i=1} f(x) = q f(p) = q p f(1) = x a
\]
\subsection*{x < 0} 
$f(0) = f(x - x) = f(x) + f(-x)$ stąd wynika że $f(x) = -f(-x)$. Korzystając z poprzedniego przypadku otrzymujemy $f(x) = -a(-x) = a x$. 

\section*{$\mathbb{R}$}
Otrzymujemy dodatkowe założenie że f jest funkcją mierzalną.
\subsection*{Ciągłość f}
Pokażmy ze f jest ciągła, najpierw w okolicy zery: $\forall \epsilon > 0, \exists \delta > 0$ taka że z $|x - 0| <\delta$ wynika $|f(x) - f(0)| < \epsilon$. Zaś $|f(x) -f(0)| =|f(x)|$.

Ustalmy teraz dowolne $ \epsilon > 0$. Weźmy odpawiadająy temu zbiór $E = f^{-1}[(-\epsilon,\epsilon)]$. Jako że f jest funkcją mierzalną i przedział jest zbiorem mierzalnym to E także jest mierzalne.

Wtedy z twierdzenie Steinhausa $\exists \delta > 0$ taka że $(-\delta, \delta) \subseteq (E - E)$.
\[
 f[(-\delta, \delta)] \subseteq f[E - E] = f[ \{ x - y : x,y \in E  \}]  = \{ f(x) - f(y) : x,y \in f^{-1}[(-\epsilon, \epsilon)]  \}    
\]
Co jest równe $\{x-y : x,y \in (-\epsilon, \epsilon)\} = (-2\epsilon, 2\epsilon) $  Czyli do naszej definicji ciągłości możemy wziąc $\delta$ wynikającą z twierdzenie Steinhouse dla przedziału $(-\epsilon \frac{1}{209},\epsilon \frac{1}{209})$ i ona na pewno wystarczy. Na razie pokazalismy tylko ciągłośc w okolicy 0. Jako że $f(x+y) = f(x) + f(y)$, to w oczywisty sposób ta ciągłość się rozszerza na całego $\mathbb{R}$. 



\subsection*{Rozwiązanie}

Ustalmy dowolnego $x_0$

Mając ciągłość wiemy że $\lim_{x \to x_0} f(x) = f(x_0)$. W szczęgólności jak weźmiemy ciąg liczb wymiernych zbiegających do $x_0$. $\lim_{q \to x_0} f(q) = \lim_{q \to x_0} a q = a x_0 = f(x_0)$.
\end{document}