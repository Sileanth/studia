
\documentclass{article}

\usepackage{./../../latex_template/tpjp}

\title{Lista 1}
\author{Łukasz Magnuszewski}

\begin{document}

\maketitle

\begin{center}
    \begin{tabular}{| c | c | c | c | c | c | c | }
        \hline
    
        1 & 2 & 3 & 4 & 5 & 6 & 7 \\

        \hline

        - & - & + & - & - & - & + \\
    
        \hline
    \end{tabular}
\end{center}

\section*{Zadanie 1}

\section*{Zadanie 3}
Wprowadźmy następujące oznaczenia:

$p \mdef$ oskarżony jest winny

$q \mdef$ oskarżony miał wspólnika

Wtedy wypowiedź oskarżyciela można zapisać jako 
\[ p \impl q \]
zaś obrońcy jako 
\[ \neg (p \impl q) \]
 co jest równoważne 
\[p \land \neg q\] 
Czyli twierdzi on że oskarżony jest winny, oraz nie miał wspólnika. 
Co raczej nie zapowiada zbyt krótkiego wyroku. $\coffee$

\section*{Zadanie 7}  
Niech $\alpha$ będzie formułą rachunku zdać. Niech $L(\alpha)$ i $P(\alpha)$ oznaczają odpowiednio liczbę lewych i prawych nawiasów w formule $\alpha$. Udownijmy że dla każdej formuły zdaniowej $L(\alpha) = P(\alpha)$.

Przeprowadźmy dowód przez indukcję strukturalną:

\subsection*{1. Zmienne zdaniowe}
Ustalmy dowolną zmienną zdaniową $p$, wtedy $L(p) = 0 = P(p)$.

\subsection*{2. $\top, \bot$}
$L(\bot) = 0 = P(\bot)$ oraz $L(\top) = 0 = P(\top)$

\subsection*{3. Negacja}
Ustalmy dowolną formułę $\alpha$ taką że $P(\alpha) = L(\alpha)$, wtedy
\[P( (\neg \alpha ) ) = P(\alpha) + 1 \ind L(\alpha) + 1 = L( (\neg \alpha ) )\]


\subsection*{4. Koniunkcja}

Ustalmy dowolne formuły $\alpha, \beta$ takie że $P(\alpha) = L(\alpha)$ oraz $P(\beta) = L(\beta)$, wtedy
\[P( ( \alpha \land \beta  ) ) = P(\alpha) + P(\beta) + 1  \ind L(\alpha) + L(\beta) + 1 = L( ( \alpha \land \beta  ) )\]

Dowód dla pozostałych przypadków(impikacja, alternatywa, równoważność) analogiczny do tego dla koniunkcji $\coffee$
\end{document}