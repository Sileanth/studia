\documentclass[11pt]{article}

\usepackage{sectsty}
\usepackage{graphicx}

\usepackage{tikz-cd}
\usepackage{polski}
% Margins
\topmargin=-0.45in
\evensidemargin=0in
\oddsidemargin=0in
\textwidth=7in
\textheight=9.0in
\headsep=0.25in

\title{ Egzamin Teoria Katagorii}
\author{ Łukasz Magnuszewski }
\date{\today}

\begin{document}
\maketitle	
\pagebreak

% Optional TOC
% \tableofcontents
% \pagebreak

%--Paper--

\section{Zadanie 1}

\subsection{Kandydat na izomorfizm}
Zdefiniujmy lambda term który odpowiada morfizmowi z $A^{1 + 1}$ do $A \times A$.
\[
  F  := \lambda f. \ <f L(1), f R(1)>  
\]
Oraz jego potencjalną odwrotność:
\[
  G := \lambda p. \ \lambda t. \ match \ t \ with \ \{L(x) \mapsto \pi_1 \ p, R(x) \mapsto \pi_2 \ p\}
\]

\subsection{$G \circ F = \lambda x. \ x$}
Wszystkie przejścia łączymy przechodniością $\beta, \eta$-redukcji.
\[
   \lambda x. \ ( \lambda p. \ \lambda t. \ match \ t \ with \ \{L(x) \mapsto \pi_1 \ p, R(x) \mapsto \pi_2 \ p\}) \ (( \lambda f. \ <f L(1), f R(1)>  )\ x)
\]
\centerline{$(\beta - abs)$}
\[
   \lambda x. \ ( \lambda p. \ \lambda t. \ match \ t \ with \ \{L(x) \mapsto \pi_1 \ p, R(x) \mapsto \pi_2 \ p\}) \ (\ <x L(1), x R(1)>  )
\]
\centerline{$(\beta - abs)$}
\[
   \lambda x. \ (  \lambda t. \ match \ t \ with \ \{L(x) \mapsto \pi_1 \ (\ <x L(1), x R(1)>  ), R(x) \mapsto \pi_2 \ (\ <x L(1), x R(1)>  )\}) 
\]
\centerline{$(\beta - proj_{1,2})$}
\[
   \lambda x. \ (  \lambda t. \ match \ t \ with \{L(x) \mapsto x L(1), R(x) \mapsto x R(1) \} ) 
\]
\centerline{(równosciowa definicja koproduktu)}
\[
  \lambda x. \ x  
\]
\subsection{$F \circ G = \lambda x. \ x$}
Wszystkie przejścia łączymy przechodniością $\beta, \eta$-redukcji.
\[
    \lambda x. \ ( \lambda f. \ <f L(1), f R(1)>  ) \ (( \lambda p. \ \lambda t. \ match \ t \ with \ \{L(x) \mapsto \pi_1 \ p, R(x) \mapsto \pi_2 \ p\} ) \ x) 
\]
\centerline{$(\beta - abs)$}
\[
    \lambda x. \ ( \lambda f. \ <f L(1), f R(1)>  ) \ ( \ \lambda t. \ match \ t \ with \ \{L(x) \mapsto \pi_1 \ x, R(x) \mapsto \pi_2 \ x\} )  
\]
\centerline{$(\beta - abs)$}
\[
    \lambda x. <( \lambda t. match \ t \ with \{L(x) \mapsto \pi_1 x, \\ R(x) \mapsto \pi_2 x\} )  L(1), ( \lambda t. \ match \ t \ with  \{L(x) \mapsto \pi_1  x, R(x) \mapsto \pi_2 x\} )  R(1)> 
\]
\centerline{$(\beta - abs)$}
\[
    \lambda x. <( match \ L(1) \ with \{L(x) \mapsto \pi_1 x, \\ R(x) \mapsto \pi_2 x\} )  , (  \ match \ R(1) \ with  \{L(x) \mapsto \pi_1  x, R(x) \mapsto \pi_2 x\} )  > 
\]
\centerline{$(\beta - match_{L,R})$}
\[
    \lambda x. \ < \pi_1 \ x, \pi_2 \ x >
\]
\centerline{(równosciowa definicja produktu)}
\[
  \lambda x. \ x  
\]

\subsection*{Izomorfizm}
Z tego że pokazaliśmy powyższe $\beta, \eta$-równosci, oraz z twierdzenia o poprawności interpretacji Rachunku Lambda, możemy stwierdzić,
 że $F \circ G = id$ oraz $G \circ F = id$ czyli $F, G$ są izomorfizmami. Co pokazuje, że $A^{1 + 1} \cong A \times A$.


\pagebreak
\section{Zadanie 2}

\subsection{Informacje z wykładu i ćwiczeń}
Przypomijmy następujące fakty z wykładu i ćwiczeń:
\begin{itemize}
    \item W kategorii $Sets_*$, singletony to jednocześnie obiekty początkowe i końcowe. 
    \item Dowolne dwa obiekty początkowe są izomorficzne, podobnie jak dwa obiekty końcowe.
  \end{itemize}

\subsection{$1 + 1$ to Obiekt początkowy}
Ustalmy dowolny obiekt $X$ w $Sets_*$ i pokażemy, że istnieje dokładnie jeden morfizm z $1 + 1$ do $X$.

\[
    \begin{tikzcd}[row sep=huge]
        & X  & \\
        1\ar[ur,"X_0"] \ar[r,"L_1", swap] & 1 + 1 \ar[u,dashed,"U" description] & 1 \ar[ul,"X_0", swap] \ar[l,"R_1"]
    \end{tikzcd}
\]


\subsubsection{Istnienie}
Korzystając z diagramu powyżej, konstruujemy morfizm $U$ z $1 + 1$ do $X$ następująco:
Skoro $1$ jest obiektem początkowym, to istnieje morfizm z $1$ do $X$, oznaczmy go $X_0$.
I wstawmy $X_0$ do diagramu, jako morfizm z $1$ do $X$, na obu ramionach. I teraz z definicji koproduktu istnieje morfizm $U$ z $1 + 1$ do $X$ taki, że $U \circ L_1 = X_0$ i $U \circ R_1 = X_0$.

\subsubsection{Unikalność}
Załóżmy, że istnieją dwa morfizmy $U_1$ i $U_2$ z $1 + 1$ do $X$. Pokażemy, że $U_1 = U_2$.


Korzystając z tego samego diagramu (Ważne jest to że istnieje dokładnie jeden morfizm z 1 do X), zauważmy że oba morfizmy muszą spełniać równość $U_1 \circ L_1 = X_0$ oraz $U_2 \circ L_1 = X_0$. Bo $1$ jest obiektem początkowym. Analogicznie dla $R_1$.

Czyli oba morfizmy sprawiają, że diagram komutuje, więc $U_1 = U_2$. Bo istnieje dokładnie jeden morfizm by diagram koproduktu komutował.

\subsection{$1 + 1 \cong 1$}
Jako że $1$, oraz $1 + 1$ są obiektami początkowymi, to są izomorficzne.
%--/Paper--

\end{document}