\documentclass{article}
\usepackage[utf8]{inputenc}
\usepackage[english, polish]{babel}
\usepackage{forloop}
\usepackage[T1]{fontenc}
\usepackage{amsfonts}
\usepackage{tikz}
\usepackage{graphicx}
\usepackage{amsmath}
\title{Zadanie 1.11.A}
\author{Łukasz Magnuszewski}
\date{\vspace{-5ex}}
%\date{} 
\begin{document}

\maketitle
\section*{Treść}
Udowodnić żu suma dowolnej (nawet nieprzeliczalnej) rodziny przedziałów na prostej, postaci $[a, b], a < b$, jest zbiorem borelowskim.

\section*{Rozwiązanie}
Niech $A \subset P(\mathbb{R})$ będzie rodziną z treści zadania. Zdefinijmy na A następującą relację:
$a \sim b \iff a \cap b \neq \emptyset$. W sposób oczywisty jest ona symetryczna oraz zwrotna. Jak weźmiemy jej przechodnie domknięcie to będzie ona dodatkowo przechodnia. Czyli będzie relacją równoważności. Oznaczmy to domknięcie jako $\smile$.

Rozważmy teraz zbiór klas abstrakcji zbiora A dla relacji $\smile$. Oznaczmy go jako $D$. A dokładniej rozważmy $C = \{ \cup x :  x \in D \}$ Zauważmy że jest to rodzina rozłącznych przedziałów na prostej.

Teraz jako że każda klasa abstrakcji D, ma przynajmniej jeden element będący przedziałem $[a, b], a < b$, czyli każdy przedział należący do C ma przynajmniej jeden punkt wymierny. Z tego faktu, oraz rozłączności elementów C, wynika przeliczalność C(Istnieje funkcja różnowartościowa z C w $\mathbb{Q}$).
Jako żę $\cup A = \cup C$ która jest przeliczalną sumą zbiorów borelowskich, czyli $\cup A$ jest zbiorem borelowskim.

\end{document}



