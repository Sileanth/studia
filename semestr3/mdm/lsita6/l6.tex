\documentclass{article}
\usepackage[utf8]{inputenc}
\usepackage[english, polish]{babel}
\usepackage{forloop}
\usepackage[T1]{fontenc}
\usepackage{amsfonts}
\usepackage{tikz}
\usepackage{graphicx}
\usepackage{amsmath}
\title{Lista 6}
\author{Łukasz Magnuszewski}
\date{\vspace{-5ex}}
%\date{} 
\begin{document}

\maketitle

\section*{Zadanie 1}

W lewej stronie rozważamy wszystkie możliwe rozmiary delegacji $\sum^n_{k=1}$. Następnie dla danego rozmiaru rozważamy wszystkie możliwe składy osobowe o tym rozmiarze ${n \choose k} $ oraz ich lidera $k$. 

Z kolei po prawej stronie, rozważamy wszystkich możliwych liderów $k$. A następnie rozważamy dla każdego pozostałego pracownika czy będzie w delegacji lub nie $2^{n-1}$.

\section*{Zadanie 2}
Rozważmy 2 rozłączne przypadki i je zsumujmy. 

\subsection*{Ciągi kończące się zerem:}
W tym przypadku żeby były jakiekolwiek ciągi spełniające warunki, to 0 musi być co najmniej tyle co 1 $l \geq k$.
Żeby nie było dwóch jedynek pod rząd, to możemy skleić każdą 1 z 0 po prawej stronie. Wtedy będziemy mieli k sklejonych $1 0$ oraz $l-k$ pojedynczych $0$. Czyli możemy teraz patrzeć że mamy ciąg długości $(l-k) + k$. I teraz wystarczy rozważyć które miejsca będą zajęte przez zera, a które przez sklejenia. Wychodzi więc $ {l \choose l -k}$. Dolna część wyrazu newtona może wyjść ujemna gdy $l < k$. I wtedy przyjmuję że wartość symbolu newtona jest równa 0.

\subsection*{Ciągi kończące się jedynką:}
    Mamy więc ustaloną ostatnią pozycję ciągu. I przedostatnia pozycja ciągu musi być 0. Czyli ten przypadek można sprowadzić do poprzedniego przypadku z k zmniejszonym o 1. Wychodzi więc ${l \choose l-k+1}$.

\subsection*{Podsumowanie:}
Oba przypadki są rozłączne więc wystarczy je dodać:
$ {l \choose l -k} + {l \choose l-k+1}$.


\section*{Zadanie 3}
Będziemy przede wszystkim korzystać z tego że na przedziale $[1, n]$ jest $\lfloor \frac{n}{a} \rfloor$ liczb podzielnych przez a. Przyjmijmy następujące oznaczenia że X, Y Z, V to odpowiednio liczby podzielne prze 2, 3, 7, 5 na przedziale $[1, n]$. Wtedy:
 \[\lfloor \frac{n}{3} \rfloor = |Y|, \lfloor \frac{n}{2} \rfloor = |X|, \lfloor \frac{n}{7} \rfloor = |Z|, \lfloor \frac{n}{5} \rfloor = |V|\]

Wtedy musimy policzyć:
 \[ |(X \cup Y) \setminus (((Z \cup V) \cap X ) \cup ((Z \cup V) \cap Y )) |\]
Z zasady włączeń i wyłączeń wychodzi:


\[ 
    \begin{aligned}
    (\lfloor \frac{n}{3} \rfloor + \lfloor \frac{n}{2} \rfloor - \lfloor \frac{n}{2*3} \rfloor )
    -(
     (\lfloor \frac{n}{2 * 5} \rfloor + \lfloor \frac{n}{2 * 7} \rfloor - \lfloor \frac{n}{2*5*7} \rfloor ) \\
     + (\lfloor \frac{n}{3 * 5} \rfloor + \lfloor \frac{n}{3 * 7} \rfloor - \lfloor \frac{n}{3*5*7} \rfloor ) \\
     - (\lfloor \frac{n}{2*5*7} \rfloor +  \lfloor \frac{n}{3*5*7} \rfloor - \lfloor \frac{n}{2*3*5*7} \rfloor)
    )
    \end{aligned}
    \]

\section*{Zadanie 4}

Liczba wszystkich permutacji n-elementowych wynosi $n!$. Wystarczy odjąć od tego te permutacje w których przynajmniej jedna z k liczb pojawia się na swoim miejscu.

\[ n! - | P_1 \cup P_2 \cup P_3 \dots P_k |\]

Możemy skorzystać z zasady włączeń i wyłączeń by obliczyć co mamy odjąć. 

\[ | P_1 \cup P_2 \cup P_3 \dots P_k | = \sum^k_{i=1} (-1)^{i+1} {k \choose i} (n-i)! \]

W i-tym kroku rozpatrujemy te permutacje które mają przynajmniej i pozycji poprawnych dla zbioru k liczb. Czyli wybieramy wszystkie możliwe i liczby, stąd $ k \choose i$ oraz permutujemy resztę, stąd $(n-i)!$.

\section*{Zadanie 6}
Przyjmuję że w tym zadania rozróżniam każdy z przedmiotów.
Oznaczmy przez $\Omega$ zbiór wszystkich rozłożeń przedmiotów. Jego moc wynosi $| \Omega | = 20^n$. 
Teraz oznaczmy przez $S_i$ zbiór tych rozłożeń w których i-ta szafa jest pusta, moc takiego zbioru wynosi $| S_i | = (20-4)^n$ Bo nie możemy nic włożyć do 4 szuflad z i-tej szafy. Z kolei $(20 - 4i)^n$ to moc zbioru tych permutacji w których i konkretnych szaf jest pustych. Odpowiedzią na to zadanie jest wzór:
\[ 20^n - |S_1 \cup S_2 \dots \cup S_5 | \]
Możemy to policzyć z zasady włączeń i wyłączeń:
\[ 20^n - \sum^5_{i=1} (-1)^(i+1) {5 \choose i} (20-4i)^n \]
W i-tym kroku rozpatrujemy wszystkie rozłożenia w których co najmniej i szaf jest pustych.


\section*{Zadanie 10}
Wszystkich ustawień gdzie na przemian siedzą kobieta i mężczyzna jest $2 (n!)^2$, dwójka się bierze stąd że musimy ustalić która parzystość miejsc jest zajmowana przez którą płeć. $n!$ to liczba jak mogą sie ustawić mężczyźni, i kobiety.

Teraz od tej liczby należy odjąć te ustawienia w których przynajmniej jedna para siedzi koło siebie. Można to policzyć z zasady włączeń i wyłączeń.

Najpierw policzy wzór na te rozłożenia w których przynajmniej k par siedzi koło siebie.
\[ P_k = 2 {n \choose k} k! (n-k)! ^2\]
2 analogicznie co w wzorze na wszystkie kombinację. Potem wybieramy wszystkie pary które muszą siedzieć, stad ${n \choose k}$, potem wybieramy ich kolejność $k!$, a na koniec ustalamy kolejność pozostałych osób $(n-k)! ^2$.
Czyli ostateczna odpowiedź to:
\[ 2 (n!)^2 - \sum^n_{k=1} P_k \]

\section*{Zadanie 9} 
Policzmy najpierw liczbę wszystkich możliwych ustawień
$(2n)!$. Teraz wystarczy odjąć wszystkie niepoprawne rozstawienia. Wyznaczmy wzór na rozstawienie w których przynajmniej k wrogów siedzi koło siebie:
\[ {n \choose k} 2^k (2n - 2k + k)! \]
Najpierw wybieramy te pary ${n \choose k}$ potem w każdej parze ustalamy która osoba jest pierwsza $2^k$, od tego momentu traktujemy taką parę jak jedną osobą. To teraz musimy policzyć wszystkie rozłożenia pozostałych osób $2n - 2k$ oraz naszych k par. Stąd wychodzi $(2n - 2k + k)!$.
\newline
Teraz wystarczy skorzystać z zasady włączeń i wyłączeń.
\[ (2n)! - \sum^n_{k=1} (-i)^{i+1} {n \choose k} 2^k (2n - 2k + k)! \]
\end{document}



