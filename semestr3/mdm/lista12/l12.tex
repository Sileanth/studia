\documentclass{article}
\usepackage[utf8]{inputenc}
\usepackage[english, polish]{babel}
\usepackage{forloop}
\usepackage[T1]{fontenc}
\usepackage{amsfonts}
\usepackage{tikz}
\usepackage{graphicx}
\usepackage{amsmath}
\usepackage{graphicx}
\usepackage{listings}
\newtheorem{theorem}{Theorem}[section]
\newtheorem{corollary}{Corollary}[theorem]
\newtheorem{lemma}[theorem]{Lemma}
\title{Lista 12}
\author{Łukasz Magnuszewski}
\date{\vspace{-5ex}}
%\date{} 
\begin{document}
\maketitle
\section*{Zadanie 12}

\subsection*{$MST(G) \leq TSP(G)$}
Zauważmy że krawędzie wchodzące w skład $TSP(G)$ uspójniają graf. 
A najmniejszy koszt uspójnienia grafu to $MST(G)$ czyli ta nierównośc zachodzi.

\subsection*{$ 2MST(G) \geq TSP(G)$}
Zauważmy że jak wybierzemy dowolny wierzchołek z $MST(G)$. I puścimy go w tym drzewie. 
To odwiedzi on każdy wierzchołek, oraz odwiedzi każda krawędź dokładnie dwa razy 
(raz w jedną, raz w drugą stronę). Czyli mamy kandydata na $TSP(G)$, 
którego koszt wynosi $2 MST(G)$. Czyli $TSP(G)$ nie może być większe.

\section*{Zadanie 1}
Puśmy następującego dfsa w dowolnym punkcie. I w zmiennej
 ans otrzymamy punkt rozpinający jeśli on istnieje.

\lstinputlisting[language=c++]{low.cpp}
Mój algorytm to pojedyńczego wywołania dfsa, więc jego złożoność to $O(n+m)$. 

A poprawnośc można uzasadnić w następujący Popatrzmy na drzewo wywołań dfs 
(Czyli wierzchołki orginalnego grafu oraz krawędzie którymi przeszedł dfs). Kluczowa obserwacja jest taka że wierzchołek może być połączony krawędzia nie drzewiową, tylko z wierzchołkiem który 
leży na ścieżce między nim a korzeniem. Bo gdyby istniała taka krawędź niedrzewiowa która nie idzie do przodka, to by dfs nią przeszedł, czyli byłaby drzewiowa sprzeczność.

Jeśli korzeń naszego drzewa ma stopień większy niż 1 to jest on punktem artykulacji. 
Udowodnijmy to: na pewno ten wierzchołek rozspójnia nasze drzewo, ale być może krawędzie które nie weszły
w skład naszego drzewa, zapobiegają temu. Ale one mogą łączyć tylko wierzchołek i jego przodka, czyli nie mogą łączyć rozłącznych podrzew.


Teraz rozpatrzmy pozostałe wierzchołki, policzmy dla nich funkcje low. Czyli jak wysoko można zajść w górę z danego wierzchołka,
mogąc iść dowolną liczbę razy krawędziami drzewowymi w dół drzewa, i maksymalnie jeden w góre krawędzią niedrzewiową.
Jeśli low danego wierzchołka jest mniejsze równe od jego głębokości to jest punktem artykulacji. 
Bo wszystkie krawędzie niedrzewiowe jego potomków, nie mogą wyjść z jego podrzewa.

\section*{Zadanie 2}
Będę rozważał tutaj grafy spójne. Ale problem dla grafu niespójnego sprowadza się odpaleniu następującego algorytmu dla każdej spójnej i wymaga by każda spójna była dwudzielna.

Graf dwudzielny można pokolorować na 2 kolory, tak by sąsiednie wierzchołki miały różne kolory. Zauważmy że z dokładnością do izomorfizmu istnieje maksymalnie jedno dwukolorowanie (To na który z dwóch kolorów pokolorujemy pierwszy wierzchołek determinuje kolorowanie całego grafu). 

\lstinputlisting[language=c++]{2col.cpp}

Algorytm sprowadza się do dfa, więc jego złożoność wynosi $O(n+m)$.

\section*{Zadanie 3} 
\lstinputlisting[language=c++]{topo.cpp}
Obserwacja jest taka że, gdy wierzchołek ma stopień wejściowy równy zero to może 
on być pierwszy w sortowaniu topologicznym. Dodatkowo możemy wtedy resztę problemu sprowadzić do grafu z usuniętym pierwszym wierzchołkiem (co zmniejsza stopień wejściowy pozostałych wierzchołków).

Złożonośc tego algorytmu wynosi $O(n+m)$. Po pierwsze trzeba wyznaczyć wierzchołki które na start mają $in(x) = 0$ kosztuje to $O(n)$ operacji. W trakcie działania programu usuniemy wszystkie krawędzie, czyli $O(m)$ operacji. I każdy wierzchołek będziemy maksymalnie raz wrzucać na listę wynikową, stąd $O(n)$ operacji. Sumarycznie daję to wymaganą złożoność.


\section*{Zadanie 6}
Załóżmy niewprost że otrzymane drzewo $T$ nie jest $MSP(G)$. weźmy wtedy minimalne $i$ takie że $e_i \in MSP(G) \land e_i \not\in T$. Wtedy istnieje $e_j$ takie że $e_j \in T \land e_j \not\in MSP(G)$, jest tak bo drzewa o takim samym rozmiarze mają tyle samo krawędzi. Zauważmy że 

\section*{Zadanie 7}
Wystarczy puścić algorytm z zadania 6 tylko dla ujemnych wag krawędzi.
\end{document}