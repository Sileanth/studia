\documentclass{article}
\usepackage[utf8]{inputenc}
\usepackage[english, polish]{babel}
\usepackage{forloop}
\usepackage[T1]{fontenc}
\usepackage{amsfonts}
\usepackage{tikz}
\usepackage{graphicx}
\title{Lista 4}
\author{Łukasz Magnuszewski}
\date{\vspace{-5ex}}
%\date{} 
\begin{document}

\maketitle

\section*{Zadanie 3}
\subsection*{Treść}
Pokaż, że iloczyn dowolnych k kolejnych liczb naturalnych dzieli się przez k!.
\subsection*{Rozwiązanie}
Iloczyn w zadaniu możemy zapisać dla pewnego $n \in \mathbb{N}$ jako:
\[\prod^{k-1}_{i = 0} (n-i)\]
Możemy domnożyć 1 zapisane jako $\frac{(n-k)!}{(n-k)!}$

\[\frac{(n-k)!}{(n-k)!} \prod^{k-1}_{i = 0} (n-i)  = \frac{1}{(n-k)!} \prod^{n}_{i=1} i  = \frac{(n)!}{(n-k)!}\]

Żeby ta liczba była podzielna przez $k!$ to iloraz tej liczby oraz k! musi być naturalny
\[\frac{(n)!}{(n-k)! k!} = {n\choose k} \in \mathbb{N} \]

\section*{Zadanie 4}
\subsection*{a)}
Przyjmijmy że w turniej gra n drużyn. Wtedy każda drużyna rozegra w trakcie całego turnieju $n-1$ meczy. Rozpatrzmy 2 przypadki:
\subsubsection*{1'}
Przynajmniej jedna drużyna rozegrała wszystkie mecze. Wtedy zagrała ona mecz z każdą inna drużyną. Czyli każda drużyna rozegrała co najmniej jeden mecz. Czyli mamy $n-1$ szufladek możliwych liczby gier: $\{1, 2, \ldots  ,(n-1)\}$. Oraz n kulek(liczba drużyn), wtedy z zasady szufladkowej przynajmniej 2 drużyny mają taką samą liczbę meczy.

\subsubsection*{2'}
Żadna drużyna nie rozegrała $n-1$ meczy. Wtedy tak samo mamy $n-1$ szufladek: $\{0,1, \ldots ,(n-2)\}$ Oraz n kulek. Czyli przynajmniej 2 drużyny rozegrały taką samą liczbę meczy.

\subsection*{b) }
Połączmy środki boków odcinkami, podzieli nam to trójkąt na 3 mniejsze trójkąty równoboczne, o bokach długości $\frac{1}{2}$. Podział ten można zobaczyć na rysunku poniżej:
\begin{center}
\begin{tikzpicture}
    \coordinate (A) at (-1, 0);
    \coordinate (B) at (1, 0);
    \coordinate (C) at (0, 1.70);
    \coordinate (D) at (-0.5, 0.85);
    \coordinate (E) at (0.5, 0.85);
    \coordinate (F) at (0, 0);
    \draw[thin] (A)--(B)--(C)--cycle;
    \draw[thin] (D)--(E);
    \draw[thin] (E)--(F)--(D);

\end{tikzpicture}
\end{center}
Naszymi szufladkami będą trójkąty (4), zaś kulkami punkty (5), wtedy z zasady szufladkowej w przynajmniej jednym trójkącie znajdują się 2 punkty. A maksymalna odległość między dwoma punktami w trójkącie to maksimum z długości boków. Jako że każdy trójkąt jest równoboczny o boku równym 0.5, to maksimum wynosi $\frac{1}{2}$. Czyli  te dwa punkty oddalone są o co najwyżej $\frac{1}{2}$.
\subsection*{c)}
Przyjmijmy że ten wielościan ma n ścian. Wtedy każda ściana ma przynajmniej 3 krawędzie i maksymalnie $n-1$ krawędzi, bo każda krawędź jest dzielona z dokładnie jedną ścianą. Więc mamy $n-1-3+1$ szufladek (możliwych liczb krawędzi), oraz n kulek (ścian). Czyli z zasady szufladkowej przynajmniej 2 ściany mają taką samą liczbę krawędzi.

\section*{Zadanie 6}
Minimalna suma maksymalnie 10 liczb ze zbioru $S$ wynosi co najmniej 0, zaś maksymalna co najwyżej 990. Daję to 991
możliwych sum. Zaś wszystkich podzbiorów $S$ jest $2^{10} = 1024$ Czyli z zasady szufladkowej wynika że przynajmniej 2 podzbiory mają taką samą sumę. 

\section*{Zadanie 10}
\subsection*{a)}
Liczba możliwych rozstawień figur szachowych na szachownicy:
\[ {64 \choose 2} {62 \choose 2} {60 \choose 2} {58 \choose 1} {57 \choose 1} {56 \choose 8} {48 \choose 8} {40 \choose 2} {38 \choose 2} {36 \choose 2} {34 \choose 1} {33 \choose 1}\]

Gdzie najpierw z wszystkich pól wybieramy białe gońce, potem z pozostałych pól czarne gońce, dalej białe wieże, białego króla, białego hetmana, białe pionki, czarne pionki, czarne wieże, białe skoczki, czarne skoczki, czarnego króla, czarnego hetmana.

\subsection*{b)}
Teraz zostało dodane ograniczenie że obaj gracze muszą mieć gońce na polach różnego koloru. Skorzystamy z poprzedniego wyniku tylko zmienimy wyrazy odpawiadające gońcom.
Najpierw wybieramy gońce białego ${32 \choose 1}$ możliwości czarnopolowego gońca, oraz
${32 \choose 1}$ białopolowego gońca. Z kolei czarny gracz ma już mniej możliwości:
${31 \choose 1}$ możliwości czarnopolowego gońca, oraz ${31 \choose 1}$  białopolowego gońca. Po podstawieniu tych liczb za 2 pierwsze wyrazy wzoru z podpunktu a wychodzi:

\[ {32 \choose 1}{32 \choose 1} {31 \choose 1} {31 \choose 1} {60 \choose 2} {58 \choose 1} {57 \choose 1} {56 \choose 8} {48 \choose 8} {40 \choose 2} {38 \choose 2} {36 \choose 2} {34 \choose 1} {33 \choose 1}\]

\subsection*{Zadanie 9}
\subsection*{a)}
\[ 10 \sum^5_{i=2} ({5 \choose i}  {9 \choose 5-i} (5-i)!)\]
Po pierwsze 10 bierze się stad że musimy rozpatrzeć wszystkie możliwości dla cyfry która występuję więcej razy. Następnie rozpatrujemy wszystkie możliwości ile ta cyfra występuje w numerze, co najmniej 2 i co najwyżej 5. Dla każdego tego przypadku musimy rozważyć które z 5 pól będą zajmowane przez tą cyfrę: $5 \choose i$. Następnie musimy rozważyć które z pozostałych cyfr pojawią się w numerze, i żadna nie może się powtórzyć, Stąd wychodzi ${9 \choose 5-i} $. Następnie trzeba ustalić w jakiej kolejności są ustawione na wolnych polach numeru, stąd $(5-i)!$.

\subsection*{b)}
Policzmy najpierw liczbę wszystkich możliwych 5 cyfrowych numerów: \[10^5\] Teraz wystarczy policzyć te numery w których nie powtarza się żadna cyfra:
\[{10 \choose 5} 5!\]
I po odjęciu otrzymamy te numery w których powtarza się co najmniej jedna cyfra
\[ 10^5 - {10 \choose 5} 5!\]


\newpage
\section*{Zadanie 8}

Liczba położonych wież musi być mniejsza równa od liczby pól na szachownicy, czyli 
\[m(k-1) + 1 \leq n m\]
Stąd wynika że $k < n \leq m$. Ponumerujmy teraz pola numerami od 0 do m, tak by w żadnym wierszu ani kolumnie nie powtarzały się numery. Można to zrobić, według wzoru $f(x,y) = (x+y) \% m$. Jest to możliwe gdyż $n \leq m$. Takie ponumerowanie gwarantuję że wieże stojące na polach o tym samym numerze się nie szachują.
Przykład dla $n = 5, m = 7$ jest na rysunku.
\begin{center}
\begin{tikzpicture}
\draw[step=1cm,gray] (-2,-2) grid (5,3);
\node at (-1.5,2.5) {0};
\node at (-0.5,2.5) {1};
\node at (0.5,2.5) {2};
\node at (1.5,2.5) {3};
\node at (2.5,2.5) {4};
\node at (3.5,2.5) {5};
\node at (4.5,2.5) {6};
\node at (-1.5,1.5) {1};
\node at (-0.5,1.5) {2};
\node at (0.5,1.5) {3};
\node at (1.5,1.5) {4};
\node at (2.5,1.5) {5};
\node at (3.5,1.5) {6};
\node at (4.5,1.5) {0};
\node at (-1.5,0.5) {2};
\node at (-0.5,0.5) {3};
\node at (0.5,0.5) {4};
\node at (1.5,0.5) {5};
\node at (2.5,0.5) {6};
\node at (3.5,0.5) {0};
\node at (4.5,0.5) {1};
\node at (-1.5,-0.5) {3};
\node at (-0.5,-0.5) {4};
\node at (0.5,-0.5) {5};
\node at (1.5,-0.5) {6};
\node at (2.5,-0.5) {0};
\node at (3.5,-0.5) {1};
\node at (4.5,-0.5) {2};
\node at (-1.5,-1.5) {4};
\node at (-0.5,-1.5) {5};
\node at (0.5,-1.5) {6};
\node at (1.5,-1.5) {0};
\node at (2.5,-1.5) {1};
\node at (3.5,-1.5) {2};
\node at (4.5,-1.5) {3};

\end{tikzpicture}
\end{center}
Naszymi szufladkami będą grupy pól o tym samym numerze, jest ich dokładnie m. Zaś naszymi kulkami będą wieże, jest ich $m(k-1) + 1$. Czyli z zasady szufladkowej w przynajmniej jednej szufladce jest k kulek. Czyli wieże odpowiadające kulkom z tej szufladki się nie szachują.
\end{document}



