
\documentclass{article}
\usepackage{./../../templatka}
\author{Łukasz Magnuszewski}

\title{Tutorial 3}

\begin{document}
\maketitle

\section*{Exercise 1}

\[
  \forall x. a(x) \implies \exists y. c(y) \land y > x 
\]


\section{Exercise 2}

$  
 \forall x, y. b(x) \land b(y) \land x < y \land (\forall z. x < z < y \implies a(z)) \land \exists O,E.
 \land (suc(x,y) \lor 
 (\forall z. x < z < y \iff z \in O \oplus z \in E.
 \land \exists xs yp. suc(x, xs) \land suc(yp, y) \land xs \in O \land ys \in E.
 \land \forall z, zs. x < z < zs < y \land suc(z, zs). z \in O \iff zs \in E \land z \in E \iff zs \in O
 ))
$


\section*{Exercise 3}
We can transform  automaton to be without epslion transitions.


Our states is powerset of old states. Initial/Accepting state is powerset of old initial/accepting states.

Transition beetwen two states is possible if all old states beloning to our state can travel somewhere. If yes, then we transfrom all old states to their all transtions. And its our new state.

\section*{Exercise 4}
definicja suc(x,y)
\[
   \forall z. z > x \implies z = y \lor z > y
\]
definicja $x < y$


$
  \exists P. x \in P \land y \in P \land (\forall z \in P. \neg (succ(z,x) \lor succ(y,z)))
  \land (\forall z \in P. x \neq z \implies \exists p. succ(p, z)) 
  \land (\forall z \in P. y \neq z \implies \exists s. succ(z, s))
$

So all 3 are eqvialent, which woudlnt be true in FSO.

\section*{Exercise 6 }
purse is easily contained in MSO 
define all 3 constructions.

Mso is definable in pure 
we just need singleton quantifiers. $kwant (x). (\forall y. y \subseteq x \implies x = y \lor y =\emptyset)$


\section*{Exercise 7}
We are doing the same proof as on lecture. But Instead $X_0, \ldots, X_{m-1}$. We define sets $B_i$ which tell us for each x, does $X_k$ corrensponding to x has i-th bit set to true($bit_i(k)$)?


\end{document}
